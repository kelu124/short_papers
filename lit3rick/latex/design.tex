\begin{minipage}{0.6\textwidth}
That said, the board is based on a HX4K/HX8K FPGA, with plenty of IOs broken out. It communicates with the outside world via an SPI bus, available either through USB with the on-board FTDI. \\

The FPGA is really at the heart of the platform, controlling pulser, ADC, TGC, interfacing with the high-speed RAM and the different IOs. \\

A benchmarking exercice showing that 50-80Msps ADCs were the norm, especially for piezos between 3 to 10MHz, I had to doubt puttign a 65Msps ADC onboard. Its 10 bits mean that I can use remaining bits on the 16-bit words to add in some information - always handy.

This platform has been used with NDT piezos, with regular piezos, and with vintage mechanical probes. It seemed to work with all - just be aware you may want to work on impedance matching sometimes, somehow.

\end{minipage}%
%
\begin{minipage}{0.4\textwidth}


        \includegraphics[width=0.95\linewidth]{images/block-diagram.png}





\end{minipage}